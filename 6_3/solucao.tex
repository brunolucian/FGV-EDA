% Created 2015-06-23 Ter 15:44
\documentclass[11pt]{article}
\usepackage[utf8]{inputenc}
\usepackage[T1]{fontenc}
\usepackage{fixltx2e}
\usepackage{graphicx}
\usepackage{longtable}
\usepackage{float}
\usepackage{wrapfig}
\usepackage{rotating}
\usepackage[normalem]{ulem}
\usepackage{amsmath}
\usepackage{textcomp}
\usepackage{marvosym}
\usepackage{wasysym}
\usepackage{amssymb}
\usepackage{hyperref}
\tolerance=1000
\usepackage{minted}
\usemintedstyle{perldoc}
\author{Alice Duarte Scarpa, Bruno Lucian Costa}
\date{2015-06-23}
\title{Exercício 6.3 (Papadimitriou)}
\hypersetup{
  pdfkeywords={},
  pdfsubject={},
  pdfcreator={Emacs 24.4.1 (Org mode 8.2.10)}}
\begin{document}

\maketitle

\section{Enunciado}
\label{sec-1}

O Yuckdonald's está considerando abrir uma cadeia de restaurantes em
Quaint Valley Highway (QVG). Os $n$ locais possíveis estão em uma
linha reta, e as distâncias desses locais até o começo da QVG são, em
milhas e em ordem crescente, $m_1, m_2, \ldots, m_n$. As restrições
são as seguintes:

\begin{itemize}
\item Em cada local, o Yuckdonald's pode abrir no máximo um
restaurante. O lucro esperado ao abrir um restaurante no local
$i$ é $p_i$, onde $p_i > 0$ e $i = 1, 2, \ldots, n$.
\item Quaisquer dois restaurantes devem estar a pelo menos $k$
  milhas de distância, onde $k$ é um inteiro positivo.
\end{itemize}

Dê um algoritmo eficiente para computar o maior lucro total
esperado, sujeito às restrições acima.

\section{Introdução}
\label{sec-2}
\label{sec-2}

Com este exercício vamos abordar uma técnica chamada de programação
dinâmica, que tem como caracteristica que a solução ótima pode ser
calculada de soluções de subproblemas.

Antes porém, vai ser apresentado uma solução utilizando um algoritmo guloso.


\section{Soluções para o problema}
\label{sec-3}
\label{sec-3}

\subsection{Algoritmo guloso}
\label{sec-3-1}

\label{sec-3-3}

Esse algoritmo recebe duas listas de tamanho $n$, uma com as distâncias
dos locais até o ponto inicial e a outra com os respectivos lucros
esperados, e um inteiro $k$ que é a distância em milhas do enunciado.
Começamos nosso algoritmo saindo do ponto inicial, em direção ao fim
da QVH.

Iremos percorrer as lojas em ordem crescente (lembrando que a entrada
já vem nessa ordem), escolhendo uma loja sempre que a distância dela
for pelo menos o valor da variável \verb~distancia_possivel~. Ao escolhermos
uma loja \verb~i~, mudamos o valor dessa variável para \verb~distancia[i] + k~,
de modo a não pegar lojas próximas dela.

\begin{minted}[]{python}
def restaurante(distancias, k, lucros):
    distancia_possivel = 0
    lucro = 0

    # Percorrendo toda a QVH
    for i in range(len(distancias)):
        if distancias[i] >= distancia_possivel:
            lucro += lucros[i]
            distancia_possivel = distancias[i] + k

    return lucro
\end{minted}


Vamos mostrar um exemplo no qual esse algoritmo não retonar o valor
máximo possivel e vamos tentar entender.


Chamada da função:

\begin{minted}[]{python}
print restaurante([3, 8, 9, 15], 3, [5, 6, 10, 8])
\end{minted}

Resultado:

\begin{verbatim}
19
\end{verbatim}

O resultado obtido utilizando desse algoritmo não foi o resultado
ótimo, pois nesse exemplo é fácil perceber que o valor máximo que se
pode ter respeitando as restrições é de $23$, no qual a escolha seria
feita pelos locais $[3, 9, 15]$. O algoritmo guloso, no entanto, está
instruído sempre a escolher o primeiro local vago respeitando as
restrições, ou seja nesse exemplo ele escolhe os locais $[3, 8, 15]$
totalizando o lucro de $19$ que é inferior ao valor ótimo. O algoritmo
guloso funcionaria bem para o caso que todos os locais têm o mesmo
lucro esperado.

Vamos resolver esse problema com utilizando um algoritmo baseado no
paradigma de programação dinâmica.

\subsection{Algoritmo utilizando programação dinâmica}
\label{sec-3-2}

Esse técnica de programação utiliza as soluções dos sub-problemas para
calcular a solução do problema.

Vamos definir o nosso sub-problema: Suponha $L(i)$ como o lucro
máximo que podemos obter com os locais de $1 \text{ até } i$ e que
$L(0)=0$. Nosso algoritmo deve seguir a seguinte regra:
\[ L(i) = \max{(L(i-1),p_i + L(i_{dispo}))} ,\]
onde $i_{dispo}$ é o maior $j$ tal que $m_j \leq m_i - k$, ou seja
o primeiro local antes de $i$ que esteja a pelo menos $k$ milhas de
distância.

Vamos usar uma função auxiliar \verb~computa_i_dispo~ para pré-computar, em
tempo linear, o valor de $i_{dispo}$ descrito acima para todo $i$.  A
função coloca $-1$ na posição $i$ do array se não há índice $j$ com
essa propriedade, isto é, se $m_i < m_1 + k$ (lembrando que os índices
começam de $1$ no enunciado, mas que os índices do código começam de
zero).

Para tornar o cálculo mais eficiente, exploramos o fato de que a
ordenação da entrada implica que $(i+1)_{dispo} \geq i_{dispo}$, isto
é, que voltar k milhas a partir da $(i+1)$-ésima casa resulta em um
índice maior ou igual que voltar $k$ milhas a partir da $i$-ésima
casa. Isso permite reusar o valor da variável \verb~k_milhas_para_tras~
entre iterações do \verb~for~.

\begin{minted}[]{python}
def calcula_i_dispo(distancias, k):
    i_dispo = []
    k_milhas_para_tras = -1

    for i in xrange(len(distancias)):
        while distancias[k_milhas_para_tras + 1] <= distancias[i] - k:
            k_milhas_para_tras += 1
        i_dispo.append(k_milhas_para_tras)

    return i_dispo
\end{minted}

O algoritmo acima parece quadrático, mas não é: O loop while interno
ou falha o teste e sai imediatamente ou incrementa a variável
\verb~k_milhas_para_tras~. A primeira coisa ocorre apenas uma vez por
iteração do \verb~for~, e portanto ocorre $n$ vezes no total. A segunda
coisa também só ocorre $n-1$ vezes, pois o maior valor possível de
\verb~k_milhas_para_tras~ é $n-2$, visto que \verb~distancias[n-1]~ não pode ser menor
que \verb~distancias[i] - k~ para nenhum $i$ pois a lista está ordenada.

Usando a função acima, podemos implementar o algoritmo de maneira simples.
Esse algoritmo recebe duas listas de tamanho $n$, uma com as distâncias
dos locais até o ponto inicial e a outra com os respectivos lucros
esperados, e um inteiro $k$ que é a distância em milhas desejada.

\begin{minted}[]{python}
def restaurante(distancias, k, valores):
    i_dispo = calcula_i_dispo(distancias, k)

    # Iniciando vetor de zeros de tamanho n+1
    lucros = [0 for _ in range(len(valores) + 1)]

    for i in range(len(distancias)):
        # Calculando L(i_dispo)
        d_novo = lucros[i_dispo[i] + 1]

        # Calculando o lucro acumulado da posicao i
        lucros[i + 1] = max(lucros[i], valores[i] + d_novo)

    return lucros[-1] # retorna o maior lucro calculado
\end{minted}

Vamos repetir o exemplo que utilizamos no algoritmo guloso e vamos
perceber que o valor que retornamos é o valor máximo.

Chamada da função:
\begin{minted}[]{python}
print restaurante([3, 8, 9, 15], 3, [5, 6, 10, 8])
\end{minted}

Resultado:

\begin{verbatim}
23
\end{verbatim}

Isso se deve ao fato de que a programação dinâmica utiliza os valores
de lucros já calculados e guardados na memória para calcular os
valores ótimos futuros.

\section{Complexidade}
\label{sec-4}
\label{sec-5}


A solução obtida pelo algoritmo guloso possui complexidade linear,
porém não é ótima, como podemos perceber no exemplo.

A solução obtida pelo algoritmo utlizando programação dinâmica é a
solução ótima e possui complexidade linear, pois a função
\verb~calcula_i_dispo~ é linear como já argumentado e a função
\verb~restaurante~ cria um array de tamanho $n+1$ e depois executa um
\verb~for~ com operações de tempo constante.
% Emacs 24.4.1 (Org mode 8.2.10)
\end{document}
